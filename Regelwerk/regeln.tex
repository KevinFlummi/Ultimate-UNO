\documentclass[a4paper,landscape, 10pt,twocolumn]{scrartcl}

\usepackage{geometry}
\geometry{left=0.75cm, right=0.75cm, top=0.25cm, bottom=1cm}
\usepackage{changepage}
\usepackage[automark]{scrlayer-scrpage}
\clearpairofpagestyles
\usepackage[utf8]{inputenc}
\usepackage{fontspec}
\usepackage{lmodern}
\usepackage{microtype}
\frenchspacing
\usepackage{setspace}
\clubpenalty = 10000
\widowpenalty = 10000
\displaywidowpenalty = 10000
\usepackage{rotating}
\usepackage[ngerman]{babel}
\selectlanguage{german}
\usepackage[babel,german=quotes]{csquotes}
\usepackage{xcolor}
\usepackage{titlesec}
\usepackage{lipsum}

\titleformat{\subsection}
    {\color{teal}\Large\vspace{-0.7em}}
    {}
    {0em}
    {\vspace{-0.3em}}
\titleformat{\subsubsection}
    {\color{teal}\large\vspace{-1.2em}}
    {}
    {0em}
    {\vspace{-0.3em}}
\def\itemspace{\vspace{6pt}\\}

\begin{document}
\twocolumn[\begin{@twocolumnfalse}
    \begin{center}
        \LARGE{\textcolor{teal}{Extreme UNO + Skiregeln - Regelwerk (Beta 0.6.1)}} \\ \vspace{5pt}
        \large{Offizielle Farbennamen: Pink, Grün, Blau, Orange \quad - \quad ''unblockbar'' > \{''schützt'', ''gibt zurück'', ''deaktiviert''\}}
        \rule{\linewidth}{1pt}
    \end{center}
\end{@twocolumnfalse}]
    \small
    \subsection{\textbf{Farbige Karten} - Nur bei gleicher Farbe spielbar}
    \begin{tabular}{p{3cm}p{9.2cm}}
        \textbf{Doge -1}:        & - Die nächste Person spielt einen Zug mehr \\
                                 & - Reduziert $\pm$X-Karten um 1 \itemspace
        \textbf{Big Chungus}:    & - Wechselt die Spielrichtung \\
                                 & - Richtet alle $\pm$X-Karten/Aussetzer in die neue Richtung\itemspace
        \textbf{Fred Perry}:     & - Ohne vorige Angriffskarte: Lässt die nächste Person aussetzen\\
                                 & - Mit voriger Angriffskarte: Schiebt alle $\pm$X-Karten weiter\itemspace
        \textbf{Crab Rave}:      & - Wechselt die Spielrichtung samt aller Angriffskarten und lässt die nächste Person aussetzen\itemspace
        \textbf{Swiper / Mopsie}:& - Tausche die Handkarten mit der entsprechenden Person \\
                                 & - \textbf{KEIN Richtungswechsel!}\itemspace
        \textbf{Monster Reborn}: & - Ziehe eine beliebige Karte vom Ablagestapel \itemspace
        \textbf{Ward}:           & - Schaue in die Karten einer beliebigen Person \itemspace
        \textbf{Illuminati}:     & - Alle legen die Karten offen \itemspace
        \textbf{Ooooof}:         & - Jede $\pm$X-Karte erhält zusätzlich +2 \\
                                 & - Endet beim Wechsel der Farbe oder sobald die Angriffsrunde beendet wurde\itemspace
        \textbf{Kondom}:         & - Schützt vor Swiper, Mopsie, Ward, Offensiven Wild Cards \itemspace
    \end{tabular}

\setlength{\parindent}{0pt}
\rule{\linewidth}{0.25pt}

    \subsection{\textbf{Action Cards} - Nur bei gleicher Farbe spielbar}
    \begin{tabular}{p{3cm}p{9.5cm}}
        \textbf{Reset}:           & - Setzt das Spiel zurück (wird nicht wieder zum Deck hinzugefügt)\itemspace
        \textbf{XBOX}:            & - Klaue den nächsten Zug einer beliebigen Person \itemspace
        \textbf{Kommunismus}:     & - Alle Handkarten zusammenlegen und kameradschaftlich aufteilen \itemspace
        \textbf{Nami (One Piece)}:& - Tausche deine Handkarten mit einer beliebigen Person \itemspace
        \textbf{Ditto}:           & - Kopiert die letzte Karte \itemspace
        \textbf{Berserker's Soul}:& - (Unblockbar) Benutze alle deine $\pm$X-Karten auf eine beliebige Person \itemspace
        \textbf{No U}:            & - Gibt $\pm$X, Aussetzer, Richtungswechsler, XBOX, Ward, No U und Offensive Wild Cards an den Angreifer zurück\\
                                  & - \textbf{Richtungswechsel!} \itemspace
        \textbf{.exe}:            & - Deaktiviert $\pm$X, Aussetzer, Richtungswechsler, XBOX, Ward, Swiper/Mopsie, Offensive Wild Cards, No U, Reset, Kommunismus und Nami \itemspace
    \end{tabular}

\newpage

    \subsection{\textbf{Wild Cards} - Auf alle Farben spielbar; Farbwünscher}
    \begin{tabular}{p{3cm}p{9.5cm}}
      \textbf{Creeper}:           &	- Ziehe sofort 3 Karten und lege die Karte ab\itemspace
      \textbf{Thanos}:            & - Wirf die eigenen Handkarten ab und ziehe halb so viele neue Karten\itemspace
      \textbf{Monopoly}:         	& - Wirf alle eigenen Karten der gewünschten Farbe ab\itemspace
      \textbf{Draw 4 (x2)}: 	    & - +4 Karten; Alle vorherigen +X Karten werden verdoppelt\itemspace
      \textbf{Draw 6 (x2)}:       & -	+6 Karten; Alle vorherigen +X Karten werden verdoppelt\itemspace
    \end{tabular}
    \subsubsection{\textbf{Offensive Wild Cards} - Unterklasse der Wild Cards}
      \begin{tabular}{p{3cm}p{9.5cm}}
        \textbf{Greed Pact}: 	      & - Alle ziehen 2 Karten\itemspace
        \textbf{Draw Die Roll}: 	  & - Die nächste Person muss würfeln und die Anzahl an Augen als Karten ziehen\itemspace
        \textbf{Thaddäus}: 	        & - Die nächste Person muss bis zur beim Legen gewünschten Farbe ziehen\itemspace
        \textbf{Catch me outside}: 	& - Die nächste Person zieht den gesamten Ablagestapel, bis auf dieser Karte\itemspace
        \textbf{Suprised Pikachu}: 	& - Die nächste Person wirft alle Wild Cards von der Hand ab\itemspace
        \textbf{Hurting Economy}: 	& - Nimm alle Wild Cards der Hand der nächsten Person\itemspace
        \textbf{360 Noscope}: 	    & - Kann an eine beliebige Person gerichtet werden; Bei ≥5 Handkarten zieht sie 10 Karten; Bei <5 Handkarten wirft sie alle Karten ab und zieht 20 neue Karten;\itemspace
        \textbf{D-D-D-D-D-Duel}: 	  & - Kann an eine beliebige Person gerichtet werden; Unblockbar; Die angreifende Person und die betroffene Person legen zeitgleich eine ihrer Handkarten vor sich; Die Person mit dem niedrigeren Kartenwert zieht 15 Karten; Bei Unentschieden ziehen beide 4 Karten\itemspace
      \end{tabular}

\setlength{\parindent}{0pt}
\rule{\linewidth}{0.25pt}

    \subsection{\textbf{Special Action Cards} - Unabhängig vom Kartenstapel}
    \begin{tabular}{p{3cm}p{9.5cm}}
      \textbf{U-NO}: 		          & - Immer Spielbar, sobald jemand „UNO“ sagt: Die Person, die „UNO“ gesagt hat, muss 2 Karten ziehen; Auch während des eigenen Zuges ablegbar, aber ohne Effekt\itemspace
      \textbf{Exodia}: 	          & - Sofortiger Gewinn, wenn man alle 5 Teile auf der Hand hat; Beim Ablegen während des eigenen Zuges: Der Ablagestapel wird mit dem Deck neu gemischt und eine neue Karte wird aufgedeckt\itemspace
    \end{tabular}

% patchnotes sind eigene Datei
\newpage
\twocolumn[\begin{@twocolumnfalse}
    \begin{center}
        \LARGE{\textcolor{teal}{\vspace{-0.25em}Patch Notes}}
        \rule{\linewidth}{1pt}
    \end{center}
\end{@twocolumnfalse}]

\subsection{Beta 0.6}
  \begin{itemize}
    \item \textbf{Präzisere Formulierungen}
    \item \textbf{Layoutänderungen}
    \item \textbf{Git Integration}
  \end{itemize}

\subsection{Beta 0.5.1}
  \begin{itemize}
    \item \textbf{Formkorrekturen}
  \end{itemize}

\subsection{Beta 0.5.0}
  \begin{itemize}
    \item \textbf{Kondom:} schützt nicht mehr vor $\pm$X-Karten, aber dafür auch vor Swiper/Mopsie \& Wards
  \end{itemize}


 \end{document}
